\documentclass{article}
\usepackage[portuguese]{babel}
\usepackage[utf8]{inputenc}
\usepackage{microtype}
\DisableLigatures{encoding = *, family = *}
\begin{document}
\title{Manual do Usuário Task Automator}
\author{Daniel Telles McGinnis}
\maketitle
\section{Introdução}
Task Automator é um programa gratuito que lhe permite agendar uma tarefa para seu computador, como exibir uma mensagem todo dia numa certa hora, ou desligar o computador daqui a duas horas.
\section{A Lista de Tarefas}
Uma vez que você adiciona uma tarefa à lista de tarefas, ela será executada automaticamento pelo Task Automator nos dias e horários especificados, sem intervenção de usuário. Adicione algumas tarefas, minimize o aplicativo para a bandeja do sistema, e relaxe.
\section{Adicionando uma Tarefa}
Adicionar uma tarefa é bem simples, mas há atalhos. Para adicionar uma tarefa, você pode:\\
\\
Clicar no botão Adicionar na barra de ferramentas\\
Clicar duas vezes na área vazia da lista de tarefas\\
Clicar com o botão direito na lista de tarefas e selecionar a opção Adicionar\\
Pressionar a tecla Insert no seu teclado\\

Você pode escolher uma das seguintes ações para as suas tarefas.\\
\\
Lembre-me (te lembra de alguma coisa com uma mensagem)\\
Visitar um site (visita um site em seu navegador padrão)\\
Abrir um arquivo (abre um arquivo no programa associado com seu tipo)\\
Rodar um programa (roda um programa, opcionalmente com argumentos)\\
Fazer logoff\\
Desligar o computador\\
Reiniciar o computador\\

\emph{Quando} a tarefa é executada é controlado por dois parâmetros, sua data e seu horário. Aqui estão as suas opções para a \emph{data}:\\
\\
Específico (a tarefa será executada na data especificada)\\
Hoje (a tarefa será executada no dia em que ela for criada)\\
Amanhã (a tarefa será executada no dia após o dia em que ela for criada)\\
Mensalmente (a tarefa será executada nos especificados dias do mês)\\
Semanalmente (a tarefa será executada nos especificados dias da semana)\\
Diariamente (a tarefa será executada todo dia)\\

Para o \emph{horário}, você pode escolher:\\
\\
Específico (a tarefa será executada na hora especificada)\\
Cronômetro (a tarefa será executada algum tempo após ser criada)\\
Repetitivo (a tarefa será executada em intervalos regulares)\\

\section{Editando uma Tarefa}
Editar tarefas também é muito fácil. Para editar uma tarefa existente, você pode:\\
\\
Selecionar a tarefa e clicar no botão Editar na barra de ferramentas\\
Clicar duas vezes na tarefa\\
Clicar na tarefa com o botão direito e selecione a opção Editar\\
Selecionar a tarefa e pressionar Enter\\

Task Automator não executa uma tarefa se ela está sendo editada, então se você começar a editar uma tarefa que deve ocorrer diariamente e chega a hora de executar a tarefa, ela será pulada aquele dia, a não ser que você edite a tarefa de tal forma que ela novamente se torne pronta a ocorrer algum tempo depois naquele dia.

\section{Removendo uma Tarefa}
Para remover uma tarefa existente, você pode:\\
\\
Selecionar a tarefa e clicar no botão Remover na barra de ferramentas\\
Clicar na tarefa com o botão direito e selecionar a opção Remover\\
Selecionar a tarefa e pressionar a tecla Delete no seu teclado
\pagebreak
\section{Configuração}
Você pode configurar os seguintes parâmetros no Task Automator.

\subsection{Língua}
Você pode escolher a língua da interface de usuário do Task Automator. As atualmente disponíveis são \emph{English (US)} e \emph{Português (Brasil)}.

\subsection{Exibir barra de ferramentas (Alt para alternar)}
Você pode alternar a visibilidade da barra de ferramentas marcando ou desmarcando essa opção, ou pressionando a tecla Alt na janela principal. Isso pode ser útil quando você prefere os atalhos (como clicar duas vezes na tarefa para editá-la ou pressionar a tecla Delete para removê-la), e nesse caso os botões da barra de ferramentas, que não seriam usados, não estariam mais visíveis.

\subsection{Alternar cor de fundo da lista de tarefas}
Esse parâmetro determina se a cor de fundo dos itens da lista de tarefas deve alternar.

\subsection{Confirmar para remover tarefas}
Esse parâmetro determina se Task Automator deve pedir confirmação para remover tarefas.

\subsection{Confirmar para sair do programa}
Esse parâmetro determina se Task Automator deve pedir confirmação quando o usuário tenta fechá-lo.

\subsection{Minimizar para a bandeja do sistema}
Esse parâmetro determina se Task Automator deve minimizar-se para a bandeja do sistema em vez de para a barra de tarefas.

\subsection{Fechar para a bandeja do sistema}
Esse parâmetro determina se Task Automator deve minimizar-se para a bandeja do sistema em vez de fechar completamente quando você clica no botão de fechamento da janela principal.

\subsection{Rodar automaticamente quando eu me logar}
Esse parâmetro determina se Task Automator deve rodar automaticamente quando você logar na conta de usuário do seu computador.
\end{document}
