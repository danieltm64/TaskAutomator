\documentclass{article}
\usepackage[utf8]{inputenc}
\usepackage{microtype}
\DisableLigatures{encoding = *, family = *}
\begin{document}
\title{Task Automator User Manual}
\author{Daniel Telles McGinnis}
\maketitle
\section{Introduction}
Task Automator is a free program that allows you to schedule your computer to do something, such as displaying a message every day at a specific time, or turning off the computer in two hours.
\section{The Task List}
Once you add a task to the task list, it will be carried out automatically by Task Automator at the specified days and times without further user intervention. Add some tasks, minimize the application to the system tray, and relax.
\section{Adding a Task}
Adding tasks is pretty straightforward, but there are shortcuts. To add a task, you can either:\\
\\
Click on the Add button on the toolbar\\
Double-click on the blank area of the task list\\
Right-click on the task list and select the Add option\\
Press the Insert key  on your keyboard\\

You can choose one of the following actions for your tasks.\\
\\
Remind me (reminds you of something with a message)\\
Visit a website (opens a website in your default web browser)\\
Open a file (opens a file in the application associated with its file type)\\
Run a program (runs a program executable, optionally with arguments)\\
Log me off\\
Shut down the computer\\
Restart the computer\\

\emph{When} the task is performed is controlled by two parameters, its date and its time. Here are your options for the \emph{date}:\\
\\
Specific (the task will be performed on the specified date)\\
Today (the task will be performed the day the task is added)\\
Tomorrow (the task will be performed the day after the day the task is added)\\
Monthly (the task will be performed every specified days of the month)\\
Weekly (the task will be performed every specified days of the week)\\
Daily (the task will be performed every day)\\

For the \emph{time}, you can choose:\\
\\
Specific (the task will be performed at the specified time)\\
Timer (the task will be performed some time after the task is added)\\
Repetitive (the task will be performed at regular intervals)\\

\section{Editing a Task}
Editing tasks is also very easy. To edit an existing task, you can either:\\
\\
Select the task and click on the Edit button on the toolbar\\
Double-click on the task\\
Right-click on the task and select the Edit option\\
Select the task and press Enter\\

Task Automator will not carry out a task if it is currently being edited, so if you start editing a task that is supposed to occur daily and it comes time to perform the task, it will be skipped that day, unless you edit the task appropriately so that it once again becomes ready to occur some time later that day.

\section{Removing a Task}
To remove an existing task, you can either:\\
\\
Select the task and click on the Remove button on the toolbar\\
Right-click on the task and select the Remove option\\
Select the task and press the Delete key on your keyboard
\pagebreak
\section{Configuration}
You can configure the following settings in Task Automator.

\subsection{Language}
You can choose the language of Task Automator's user interface. The ones currently available are \emph{English (US)} and \emph{Português (Brasil)}.

\subsection{Display toolbar (Alt to toggle)}
You can toggle the visibility of the toolbar by checking or unchecking this option, or by pressing the Alt key in the main window. This can be useful when you prefer the shortcuts (such as double-clicking a task to edit it or pressing the Delete key to remove it), in which case the toolbar buttons that would go unused would no longer be there.

\subsection{Alternate background color of task list}
This setting determines whether the background color of the task list's entries should alternate.

\subsection{Confirm to remove tasks}
This setting determines whether Task Automator should ask for confirmation when removing tasks.

\subsection{Confirm to exit the program}
This setting determines whether Task Automator should ask for confirmation when the user tries to close it.

\subsection{Minimize to system tray}
This setting determines whether Task Automator should minimize itself to the system tray instead of the taskbar.

\subsection{Close to system tray}
This setting determines whether Task Automator should minimize itself to the system tray instead of ending completely when you click the close button of the main window.

\subsection{Start automatically when I log in}
This setting determines whether Task Automator should start automatically when you log into your computer's user account.
\end{document}
